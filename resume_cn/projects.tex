\cvsection{项目和竞赛经历}

\begin{projitem}
    {用强化学习的方法解出2018年数模国赛B题}
    {团队负责人}
    {2018年10月}{2018年10月 }
    \item 用\textbf{深度强化学习}的算法(DQN)解决工厂CNC机器的排班自动化问题
    \item 为团队编写论文并排版
    \item 得到了接近最优解(参考答案给出)的结果
\end {projitem}
\begin{projitem}
    {大学生创新创业实践项目(国创)}
    {项目负责人}
    {2017年10月}{2018年11月}
    \item 尝试用数学的方法设计一个社交软件使得它的社交网络活跃程度尽可能地大.
    \item 对社交网络基于 \textbf{常微分方程} 进行建模.
    \item 用大量的\textbf{数值方法}解出结果.
    \item 项目被学校评为\textbf{国家级大学生创新创业实践项目},并顺利结题.
\end{projitem}
\begin{projitem}
    {协助编辑了一本关于抽象代数的教材}
    {协作者}
    {2018年10月}{2019年8月}
    \item 为华东师范大学数学科学学院的杜荣教授整理并编辑了他所讲授的\textbf{《近世代数II》}(有关于Galois理论的本科生抽象代数课程)的讲义,使得其能够成为日后的教材。
    \item 对讲义内容(包括旦不限于 介绍文字、插图、定理证明、例题解法等)做了一些微小的补充,使得其内容更容易为本科生读者所理解,且更加连贯。
    \item 成书逾一万两千字,约有20个章节。
\end{projitem}

\endinput